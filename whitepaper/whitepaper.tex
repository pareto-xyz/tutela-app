%
% File acl2020.tex
%
%% Based on the style files for ACL 2020, which were
%% Based on the style files for ACL 2018, NAACL 2018/19, which were
%% Based on the style files for ACL-2015, with some improvements
%%  taken from the NAACL-2016 style
%% Based on the style files for ACL-2014, which were, in turn,
%% based on ACL-2013, ACL-2012, ACL-2011, ACL-2010, ACL-IJCNLP-2009,
%% EACL-2009, IJCNLP-2008...
%% Based on the style files for EACL 2006 by 
%%e.agirre@ehu.es or Sergi.Balari@uab.es
%% and that of ACL 08 by Joakim Nivre and Noah Smith

\documentclass[11pt,a4paper]{article}
\usepackage[hyperref]{acl2020}
\usepackage{times}
\usepackage{latexsym}
\renewcommand{\UrlFont}{\ttfamily\small}

% This is not strictly necessary, and may be commented out,
% but it will improve the layout of the manuscript,
% and will typically save some space.
\usepackage{microtype}

\aclfinalcopy % Uncomment this line for the final submission
%\def\aclpaperid{***} %  Enter the acl Paper ID here

%\setlength\titlebox{5cm}
% You can expand the titlebox if you need extra space
% to show all the authors. Please do not make the titlebox
% smaller than 5cm (the original size); we will check this
% in the camera-ready version and ask you to change it back.

\newcommand\BibTeX{B\textsc{ib}\TeX}

\title{Tutela: An Anonymity Tool for Ethereum and Tornado Cash}

\author{
  \small{Mike Wu, Will McTighe, Kaili Wang}\\
  \small{Stanford University} \\ \And
  \small{Nick Bax} \\
  \small{Convex Research} \\ \And
  \small{Istv\'{a}n A. Seres} \\
  \small{E\"{o}tv\"{o}s Lor\'{a}nd University} \\ \AND
  \small{Frederico Carrone, Tomas De Mattey, Manuel Puebla, Herman O. Demaestri, Mariano Nicolini, Pedro Fontana} \\
  \small{LambdaClass} \\}
\date{}

\begin{document}
\maketitle
\begin{abstract}
A common misconception among blockchain users is that decentralization guarantees privacy. The reality is almost the opposite as every transaction one makes, being recorded on a public ledger, reveals information about one's identity.
Mixers, such as Tornado Cash, were developed to preserve privacy through ``mixing'' transactions in an anonymity pool, such that it is impossible to identify who withdrew tokens from the pool. Unfortunately, it is still possible to reveal information about those in the anonymity pool if users are not careful.
We introduce Tutela, an application build on expert heuristics to report the true anonymity of an Ethereum address.
In particular, Tutela has two functionalities: first, it clusters together Ethereum addresses based on interaction history such that for an Ethereum address, we can identify other addresses likely owned by the same entity; second, Tutela computes the true size of the anonymity pool of Tornado Cash mixing, taking into account compromised addresses that reveal identity information. A public implementation of Tutela can be found at \url{https://github.com/TutelaLabs/tutela-app}.
\end{abstract}

\section{Introduction}

TODO

\section{Background}

TODO

\section{Tutela Overview}

TODO

\section{Data and Setup}

TODO

\section{Ethereum Heuristics}

\subsection{Deposit Address Reuse}

\subsection{Diff2Vec}

\section{Tornado Cash Heuristics}

\subsection{Address Match}

\subsection{Unique Gas Price}

\subsection{Multiple Denomination}

\subsection{Interaction History}

\section{History and Roadmap}

\section{Discussion}

\subsection{Limitations}

\subsection{Extensions}

\subsection{Broader Impact}

\end{document}
